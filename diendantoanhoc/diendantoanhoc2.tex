\thispagestyle{diendandayvahoctoannone}
\pagestyle{diendandayvahoctoan}
\everymath{\color{diendantoanhoc}}
\graphicspath{{../diendantoanhoc/pic/}}
%\blfootnote{$^{1}$\color[named]{diendantoanhoc}...}
\begingroup
\AddToShipoutPicture*{\put(0,616){\includegraphics[width=19.3cm]{../bannerdiendan}}}
\AddToShipoutPicture*{\put(84,525){\includegraphics[scale=1]{../tieude.pdf}}}
\centering
\endgroup
\vspace*{180pt}

\begin{multicols}{2}
	Trong bài viết này, tôi muốn nhấn mạnh việc giảng dạy Tin học và Công nghệ Thông tin (CNTT) hiện nay còn lạc hậu và điều này hạn chế việc phát triển ngành CNTT của nước ta.  
	\vskip 0.05cm
	$\pmb{1.}$ \textbf{\color{diendantoanhoc}Sự cần thiết của tư duy Tin học}
	\vskip 0.05cm
	Trong một nền kinh tế dựa vào công nghệ cao -- khi mà ``nội dung" sẽ là một trong những mặt hàng chính thì Lập trình -- thao tác trên các dữ liệu thô, tạo ra các tri thức có giá trị trở thành một kỹ năng có nhu cầu cao, một nghề nghiệp thu hút trong thị trường lao động. Gần như mọi khía cạnh của nền kinh tế, mọi hoạt động khoa học ngày nay đều có sự hiện diện của Tin học. Chẳng hạn, ngày nay chúng ta đang ``bơi trong biển dữ liệu", làm thế nào để có thể ``chắt lọc" được từ đó những tri thức đáng giá -- ``xử lý dữ liệu lớn" là minh chứng rõ nét cho nhu cầu này.
	\vskip 0.05cm
	Chính vì thế, có được một tư duy tốt về thuật toán và kỹ năng lập trình sẽ giúp ích rất lớn cho học sinh muốn tham gia vào các lĩnh vực khoa học -- công nghệ.
	\vskip 0.05cm
	Có thể nói, Tin học rèn dũa cho học sinh tư duy thuật toán và tính thực tế. Ở mức độ đơn giản, học sinh có được \textit{kỹ năng giải quyết vấn đề}: từ một tập hợp đầu vào, cần phải xử lý như thế nào để có được một kết quả thỏa mãn. Ở mức cao hơn, học sinh được rèn luyện \textit{kỹ năng sáng tạo ra vấn đề}, rồi tìm cách giải quyết nó. Ngoài ra, Tin học sẽ giúp học sinh có được \textit{sản phẩm} thông qua quá trình lao động. Điều này khiến cho việc học nói chung không chỉ mang tính lý thuyết, mà còn tăng tính thực hành.
	\vskip 0.05cm
	Qua quá trình giảng dạy, tôi nhận thấy việc dạy các kỹ năng lập trình cơ bản sẽ giúp học sinh:
	\vskip 0.05cm
	$\circ$	Tăng tốc quá trình phát triển;
	\vskip 0.05cm
	$\circ$	Thúc đẩy sáng tạo;
	\vskip 0.05cm
	$\circ$	Tăng tính tự tin;
	\vskip 0.05cm
	$\circ$	Năng cao kỹ năng giải quyết vấn đề và tư duy phản biện;
	\vskip 0.05cm
	$\circ$	Thấy được ứng dụng cụ thể của các môn khoa học khác, đặc biệt là Toán học
	\vskip 0.05cm
	$\circ$	Định hướng nghề nghiệp.
	\vskip 0.05cm
	Ở Hoa Kỳ, Cựu Tổng thống Barack Obama trong thông điệp liên bang $2013$, đã nhấn mạnh vào ``việc xây dựng các kỹ năng cho học sinh đáp ứng một nền kinh tế công nghệ cao", và sau này, ông kêu gọi giới trẻ ``thay vì chỉ biết tiêu thụ, hãy sản xuất ra thông tin", và ``không chỉ sử dụng máy điện thoại di động, hãy lập trình cho nó". Hoa Kỳ đã có nhiều chương trình tài trợ đưa việc giảng dạy lập trình vào khối tiểu học và trung học.
	\vskip 0.05cm
	Anh Quốc là quốc gia đầu tiên trên thế giới đã đưa việc học lập trình thành điều bắt buộc trong các trường tiểu học và trung học. Trẻ em sẽ học lập trình ở độ tuổi $5$ đến $16$. Ở giai đoạn $1$, học sinh học viết chương trình nhỏ, các khía cạnh đơn giản của thuật toán, cài đặt và thực thi trên thiết bị điện tử. Trong giai đoạn $2$, học sinh được học cách thiết kế và viết các chương trình phức tạp hơn, tương tác với môi trường xung quanh. Ở giai đoạn $3$ (cấp trung học), học sinh học về đại số Boolean, tư duy thuật toán. Giai đoạn $4$ tập trung vào sáng tạo và định hướng nghề nghiệp.
	\vskip 0.05cm
	$\pmb{2.}$ \textbf{\color{diendantoanhoc}Học Tin học quá muộn sẽ làm ngành CNTT tụt hậu}
	\vskip 0.05cm
	Ở Việt Nam, CNTT là một ngành quan trọng, được Đảng và Nhà nước coi là một mũi nhọn trong việc phát triển kinh tế. Và để có thể cạnh tranh trong ngành công nghiệp, chúng ta cần thật nhiều những chuyên gia -- lập trình, thiết kế hệ thống, quản trị hệ thống. Những người này phải có khả năng tương tác, nhận và bàn giao công việc với các đối tác nước ngoài. Ngành CNTT của chúng ta thực sự thiếu trầm trọng những nhân lực có chất lượng cao.
	\vskip 0.05cm
	Ở Khoa CNTT thuộc Đại học Công nghệ, ĐHQGHN, một trong những điều chúng tôi cố gắng làm, là đưa việc giảng dạy chuyên môn vào ngay từ đầu, nhưng phải đến sau năm thứ ba và thứ tư, sinh viên mới dần có đủ kỹ năng căn bản để đi thực tập. Mặc dù cấp THPT đã có môn lập trình, nhưng gần như sinh viên lên Đại học lại học lại từ đầu.
	\vskip 0.05cm
	Một lý do có lẽ là, hiện nay, Tin học ở cấp THPT là môn phụ, không có mặt trong các kỳ thi quan trọng như Kỳ thi tốt nghiệp THPT hay thậm chí trong cả các kỳ thì tuyển vào ngành CNTT ở các trường đại học. Chính vì thế, nên ở cấp THPT, việc dạy và học Tin học có phần bị buông lỏng. Học sinh không học đủ, không làm được những bài tập lập trình rất cơ bản. Giáo viên Tin học có thể phải kiêm nhiệm thêm hàng núi các công việc ``vô danh" khác và ít có điều kiện trau dồi chuyên môn. Phụ huynh nói chung cũng không quan tâm cho con con em mình học Tin học sớm.
	\vskip 0.05cm
	Trong bất kỳ lĩnh vực nào, để trở thành chuyên gia xuất sắc, bạn phải bỏ ra $10{.}000$ giờ luyện tập. Ngành CNTT cũng không ngoại lệ. Trong lĩnh vực CNTT, rất nhiều hãng khởi nghiệp bởi những thanh niên còn rất trẻ. Và những người rất trẻ này thường được tiếp xúc với máy tính, lập trình từ bé. Nhưng thực tế ở ngay Đại học Công nghệ cho thấy, chỉ có khoảng $1/10$ số sinh viên vào thẳng của Khoa CNTT, Đại học Công nghệ là có nền tảng lập trình tốt (phần lớn là học sinh đạt giải trong kỳ thi HSG Quốc gia môn Tin học). Đa phần trong số $9/10$ sinh viên còn lại dù thi Đại học ở khối A và A$1$ nhưng gần như không biết gì về lập trình, và phải học từ đầu. 
	\vskip 0.05cm
	\textbf{\color{diendantoanhoc}Lời kết}
	\vskip 0.05cm
	CNTT là ngành phụ thuộc rất lớn vào trí tuệ và kỹ năng lao động (kỹ năng lập trình, kỹ năng vận hành các hệ thống tin học). Như vậy ngành CNTT của Việt Nam hoàn toàn có thể ``cất cánh" trở thành một ngành mũi nhọn. Và để tiềm năng này trở thành hiện thực, hãy coi trọng và khích lệ học sinh học Tin học từ bé.
	\vskip 0.05cm
	Việc dạy tin học và lập trình sớm rất cần cho cho toàn bộ học sinh phổ thông, vì ngoài những người sẽ làm nghề lập trình, hầu hết người lao động ở mọi ngành nghề sẽ cần dùng tin học như một công cụ lao động cơ~bản. 
	\vskip 0.05cm
	Hệ thống Giáo dục phổ thông ngày nay đã quá lạc hậu trong việc giảng dạy CNTT. Và tôi kỳ vọng vào công cuộc Đổi mới Giáo dục mà Bộ Giáo dục đang tiến hành có thể góp phần thay đổi thực trạng này. Tuy nhiên với cách triển khai của Bộ Giáo dục, tôi có cảm nhận rằng, môn Tin học sẽ vẫn chỉ là một môn phụ. Nên chăng, các gia đình, các bạn học sinh, hãy tự trang bị cho mình công nghệ, tư duy hiện đại, để vững bước vào kỷ nguyên~số.
\end{multicols}